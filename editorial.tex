% Este fichero es parte del Número 5 de la Revista Occam's Razor
% Revista Occam's Razor Número 5
% (c)  2010 The Occam's Razor Team
% Esta obra está bajo una licencia Reconocimiento 2.5 España de Creative
% Commons. Para ver una copia de esta licencia, visite 
% http://creativecommons.org/licenses/by/2.5/es/
% o envie una carta a Creative Commons, 171 Second Street, Suite 300, 
% San Francisco, California 94105, USA.
% 
% Adaptada en el año 2017 para el Grupo GNU Linux Universidad Distrital
%

\rput(8,-1.6){\resizebox{!}{5cm}{{\epsfbox{images/general/editorial.eps}}}}
\rput(0.0,-13){\resizebox{7cm}{35.0cm}{{\epsfbox{images/general/bar.eps}}}}

\rput(0.7,-26.5){\resizebox{!}{0.9cm}{{\epsfbox{images/general/licencia.eps}}}}


\begin{textblock}{9.2}(3,0)
\begin{flushright}
{\resizebox{!}{1cm}{\textsc{Editorial}}}

\vspace{2mm}

Grupo GNU Linux \\Universidad Distrital\\Francisco José de Caldas
\end{flushright}
\end{textblock}

\vspace{4mm}

\definecolor{barcolor}{rgb}{0.9,0.9,0.9}

\begin{textblock}{30}(-1.5, -1)
\begin{minipage}{0.12\linewidth}
\sf\color{barcolor}
\begin{center}

\vspace{1cm}

\colorbox{black}{
{\resizebox{3cm}{0.7cm}{\textcolor{white}{\bf\sf\large GLUD}}}
}
{\resizebox{2.5cm}{0.4cm}{\textcolor{white}{\bf\sf\large Magazine}}}
%{\resizebox{2.5cm}{0.4cm}{\bf\sf\large Razor}}
\vspace{4mm}

{\bf Número 2.\\ Octubre 2017}

%\vspace{5cm}
\vspace{2cm}
\hrule

\vspace{7mm}
%{\bf Dirección: }
%\vspace{1mm}
%Colocar Dirección de la revista
%\vspace{2mm}
%{\bf Editores:}
%\vspace{1mm}
%Colocar comite editorial.
%\vspace{4mm}

{\bf Colaboradores:}\\
\vspace{1mm}
{\tt
José Noé Poveda\\Diego Osorio\\
Steven Sierra\\
Andrés Acosta\\
Laydi Bautista\\
Fernando Pineda\\
Marlon Cárdenas\\
Sebastian Sanchez\\
Jesus David Romero\\
Alejandro Cortazar\\
Jorge Ulises Useche\\
Juan Carlos Velandía\\
Leidy Marcela Aldana}

\vspace{15.0mm}

%{\bf Maquetación y Grafismo}

%\vspace{1mm}

%Completar

\vspace{2mm}

\hrule

\vspace{4mm}

{\bf Publicidad}

\vspace{3mm}

Occam's Razor Direct

{\tt occams-razor@uvigo.es}

\vspace{4mm}

\hrule

\vspace{6mm}

{\bf Impresión}

Por ahora tu mismo\ldots Si te apetece

\vspace{3mm}

\hrule

\vspace{9mm}

\textcopyleft 2017 GLUD Magazine \\

Esta obra está bajo una licencia Reconocimiento 2.5 España de Creative
Commons. Para ver una copia de esta licencia, visite 

{\scriptsize \href{http://creativecommons.org/licenses/by/2.5/es/}{creativecommons.org}} 

\end{center}
\end{minipage}

\end{textblock}

\begin{textblock}{20}(3,2.0)

\begin{minipage}{.45\linewidth}
\colorbox{introcolor}{
\begin{minipage}{1\linewidth}

{{\resizebox{!}{1.0cm}{E}}{l Grupo GNU Linux Universidad Distrital agradece a cada uno de los lectores de esta revista, y a los desarrolladores del c\'odigo fuente de la misma, es decir, a los encargados de la revista Occam's Razor. 


\bigskip

}
}

\end{minipage}
}

\vspace{8mm}

El Grupo GNU Linux es un Grupo Lider en la apropiación, desarrollo, uso y difusión de tecnología, ciencia y cultura libre; el cual, hace parte de la Universidad Distrital Francisco José de Caldas.

{\large 
	\begin{entradilla}
    \begin{center}
    	{\em Piensa libre, vive libre,\\El conocimiento te hace libre.}
    \end{center}    
    \end{entradilla}
}
\vspace{2mm}
GLUD Magazine fue elaborada por primera vez en el año 2011, seis años después, para este segundo volumen publica su segundo número para compartir artículos acerca de ciencia, tecnología  y cultura libre. Como grupo, esperamos que sea agradable para usted leer esta revista y que sirva como aporte a la comunidad.

\bigskip
%\medskip
\begin{flushright}
{\Large\sc{Grupo GLUD.}}
\end{flushright}
\begin{center}
\resizebox{3cm}{!}{{\epsfbox{images/general/glud.eps}}}
\end{center}

\end{minipage}

\bigskip

\colorbox{introcolor}{
\begin{minipage}{.45\linewidth}

\bigskip

{\footnotesize\sf{\color{white}
Las opiniones expresadas en los artículos, así como los contenidos de los mismos, son responsabilidad de los autores de éstos. Puede obtener la versión electrónica de esta publicación, por medio de la página:\\
\begin{center}
	{\tt \href{https://glud.udistrital.edu.co/}{glud.udistrital.edu.co}}
\end{center}

Así como el {\em código fuente} de la misma y los distintos ficheros de datos asociados a cada artículo en el repositorio de la revista en Github:\\

\begin{center}
	
{\tt \href{https://github.com/GLUD/GLUD_Magazine/}{github.com}}
\end{center}

}}

\bigskip

\end{minipage}
}


\end{textblock}

\pagebreak