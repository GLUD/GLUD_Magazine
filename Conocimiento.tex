% Este fichero es parte del Número 5 de la Revista Occam's Razor
% Revista Occam's Razor Número 5
% (c)  2010 The Occam's Razor Team
% Esta obra está bajo una licencia Reconocimiento 2.5 España de Creative
% Commons. Para ver una copia de esta licencia, visite 
% http://creativecommons.org/licenses/by/2.5/es/
% o envie una carta a Creative Commons, 171 Second Street, Suite 300, 
% San Francisco, California 94105, USA.
\rput(2.5,-2.3){\resizebox{!}{3.2cm}{{\includegraphics[]{images/general/white}}}}
\begin{flushright}
\msection{introcolor}{black}{0.35}{Utilización Software Libre}

\mtitle{10cm}{Sobre el conocimiento y la utilización del software libre en la educación superior en Bogotá}

\msubtitle{8cm}{TituloOn the knowledge and use of free software in higher education in Bogota}

{\sf José Noé Poveda Zafra, Msg. en Teleinformática}

{\psset{linecolor=black,linestyle=dotted}\psline(-12,0)}
\end{flushright}

\vspace{2mm}

\begin{multicols}{2}

\sectiontext{white}{black}{Resumen}

\intro{introcolor}{E}{ste artículo muestra un estudio sobre la penetración del software libre en las instituciones de educación superior del distrito capital de Bogotá, realizado en el mes de noviembre de 2016. El estudio tiene como objeto conocer el grado de conocimiento y uso del software libre de los estudiantes universitarios y en especial los de las universidades públicas; dada la coyuntura de las políticas del distrito en materia de inclusión de tecnologías abiertas orientadas al software.  Esta iniciativa fue realizada por el coordinador del grupo de software libre de la Universidad Distrital Francisco José de Calda y director del semillero de tecnologas TECLIBRE. El estudio está orientado en dos etapas: la primera da a conocer si los encuestados tienen conocimientos primarios del software libre y la segunda muestra, además del conocimiento sobre el tema en mención, las preferencias en la utilización de aplicaciones de software libre. Hay estudios leves sobre el uso de software libre en Bogotá, direccionado a entidades privadas. Este estudio es el primero realizado en los claustros académicos..}

\sectiontext{white}{black}{Palabras clave: } Software libre, software propietario, GNU, Bogotá, encuesta.

\sectiontext{white}{black}{I. Introducción}

La diversidad en la tecnología ha orientado a una sociedad más consciente y abierta al aprendizaje de las nuevas formas de aplicación, desarrollo y optimización de las mismas. Entre los más influyentes e importantes avances a nivel informático se encuentra el software libre, el cual ha sido utilizado por diferentes usuarios, los cuales no buscan solo adaptasen a la aplicación, sino poderla modificar y adaptarla a sus necesidades sin que haya ningún tipo de restricciones, esto da al usuario una sensación de libertar, hasta el punto de modificar el código y darle a la aplicación su sello personal, sin desconocer a los autores de la aplicación. \\

Definir al software libre es muy complejo, dado que hay diferentes matices de acuerdo a su desarrollo en el tiempo, pero la definición más tradicional es la que está relacionada con los principios filosóficos que la enmarcan, en torno a las libertades que proporciona al usuario final, tales como lo son: ejecutar los programas sin restricciones, tener acceso al código y modificarlo a la necesidad del usuario, redistribuir copias, mejorar los programas en beneficio de las comunidades. (Stallman, 2004, p. 59).\\

En estos tiempos cada día crece la aceptación del software libre SL, ya que existen aplicaciones para todas las áreas que representan beneficios para las personas, empresas y para el sector público. Las entidades que más se lucran con el SL son las empresas tecnológicas ya que es óptimo para dar valor agregado a sus productos puesto que no tienen restricciones de licencias. Las entidades educativas tienen una gran oportunidad en el SL, no solo por licencias con gratuidad o de menor costo, sino porque hay un gran desarrollo de software para la investigación y el desarrollo bajo unas premisas alrededor de la cultura de código abierto como conceptos de la libertad, la meritocracia y la colaboración.\\

Uno de los ejes fundamentales en la utilización, desarrollo y creación de aplicaciones de software de código abierto, se encuentra en las instituciones educativas y en especial las instituciones de educación superior. Los grupos de software libre, en las instituciones universitarias, hacen esfuerzos para difundir, ense\~nar y hacer proyectos con elementos de tecnologías libres, y generalmente al no tener apoyo institucional, son esfuerzos menores que hace muy lento el proceso de masificación.\\

\begin{entradilla}
Los grupos de {\color{introcolor}{Software Libre}}, en las instituciones universitarias, hacen esfuerzos para difundir, ense\~nar y hacer proyectos con herramientas de tecnologías libres.
\end{entradilla}

% Siguiente página
\ebOpage{introcolor}{0.35}{Utilización Software Libre}

Dadas las condiciones del bajo uso del software libre, se tuvo la motivación de hacer un estudio para conocer la realidad del uso de dicho software, y a través del grupo de tecnologías libres y estudiantes de la Universidad Distrital Francisco José de Caldas se dise\~nó una encuesta y se aplicó a la población objetivo. 

Esto con el propósito de realizar algunas acciones con referencia a la difusión de estas herramientas.\\

\sectiontext{white}{black}{II. Metodología}

\sectiontext{white}{black}{Dise\~no de la encuesta.}

Las encuestas est\'an divididas en tres partes; en la primera parte el encuestado debe llenar los datos b\'asicos: Nombre, edad, instituci\'on, \'area de estudio; en la segunda parte se debe llenar seg\'un el enunciado la opci\'on que considere correcta en cuanto al conocimiento que se tiene sobre SL; finalmente en la tercera parte el encuestado debe seleccionar el software que m\'as usa para las aplicaciones b\'asicas y populares, incluyendo el software propietario.\\

\sectiontext{white}{black}{Poblaci\'on objetivo}

La poblaci\'on est\'a compuesta por estudiantes de las  instituciones de educaci\'on superior y en una de educaci\'on tecnol\'ogica, tales como: Universidad Nacional de Colombia, Pontificia Universidad Javeriana, Universidad Distrital Francisco Jos\'e de Caldas y Servicio Nacional de Aprendizaje (SENA) respectivamente. \\

Para el estudio se encuestaron a 1062 estudiantes que se encuentran estudiando en las mencionadas instituciones de educaci\'on superior de Bogot\'a. Las encuestas se realizaron en el mes de noviembre de 2016 en cada una de las instituciones mencionadas.
La raz\'on por la que se escogieron estas instituciones, fue por la relevancia que tienen en la regi\'on y en el Distrital Capital de Bogot\'a, adem\'as de la inmersi\'on de sus egresados en diferentes campos muy relacionados con la tecnolog\'ia.\\

Las encuestas se hicieron persona a persona de forma 	aleatoria y voluntaria sin recibir ning\'un incentivo. La cual se dejaron registros f\'isicos de cada una de las encuestas realizadas. La muestra aqu\'i expuesta es tomada en base a las encuestas realizadas a los estudiantes de las instituciones mencionadas, con el fin de saber las expectativas en el \'ambito del software libre. Las muestras se tomaron a estudiantes de diferentes \'areas del conocimiento en cada instituci\'on, Los encuestados son hombres y mujeres de cualquier edad, religi\'on o raza, etc.; la condici\'on esencial es que sean estudiantes y pertenezcan a la instituci\'on educativa donde fue tomada.\\

El tema principal de la encuesta es el Software Libre, espec\'ificamente sobre el conocimiento y su uso.
Procedimiento en la selecci\'on de la muestra
En cada instituci\'on se abord\'o a estudiante de forma aleatoria, al cual se le hizo una corta introducci\'on sobre el objeto de dicha muestra. El estudiante que acepto la encuesta, la diligencio de acuerdo a los siguientes \'items:
	En la primera parte, se debe llenar los datos b\'asicos: Nombre, edad, instituci\'on, \'area de estudio.\\
	Como segunda parte, contest\'o un test sobre el conocimiento del SL.\\
	Y por \'ultimo selecciono una de las alternativas del software que utiliza.\\
    
\sectiontext{white}{black}{Caracter\'isticas T\'ecnicas de la encuesta}

\begin{itemize}

\item Tipo de Encuesta: Cuestionario estructurado, realizado de forma presencial y en f\'isico.

\item Muestreo: Aleatorio entre estudiantes de distintas instituciones de educaci\'on superior, que a continuaci\'on se nombran: Universidad Distrital Francisco Jos\'e de Caldas, Universidad Nacional de Colombia, Pontificia Universidad Javeriana, Universidad Pedag\'ogica de Colombia y el Servicio Nacional de Aprendizaje (SENA).

\item Tama\~no de la Muestra: 1062 entre hombres y mujeres.

\item Número de preguntas: 15

\end{itemize}

Las muestras se tabularon en una hoja de c\'alculo y se hizo la discriminaci\'on por instituci\'on educativa.\\

\vspace{2cm}
\sectiontext{white}{black}{III. Resultados y análisis}

Las muestras se concentraron en una base de datos y se procedi\'o a evaluar los valores estad\'isticos requeridos para obtener la informaci\'on, los cuales consisten en los valores porcentuales de las alternativas seleccionadas por los encuestado y que se realizaron para cada \'item. Y arrojaron los siguientes datos para cada \'item. \\

\sectiontext{white}{black}{Conocimiento sobre el software libre}

En esta primera fase de la encuesta sobre el conocimiento de software libre, se discrimino por instituci\'on para valorar la penetraci\'on del software libre en las diferentes instituciones. En esta primera fase se tienen 10 \'items, los cuales se analizar\'an por separado. Los \'items del 5 al 10 tratan sobre los principios  filos\'oficos del SL.\\
En el primer \'item,  a la pregunta: ¿ha escuchado sobre software libre?

% Siguiente página
\ebOpage{introcolor}{0.35}{Utilización Software Libre}

\begin{center}
\resizebox{8cm}{!}{{\epsfbox{images/Conocimiento/fig1.eps}}}
\mycaption{Respuestas a la pregunta: ¿ha escuchado sobre software libre?}
\end{center}

En tres de las cinco instituciones encuestadas, predomina que s\'i conocen de software libre. De los estudiantes encuestados del SENA el “No” domina significativamente respecto a las otras instituciones dado que es una instituci\'on tecnolog\'ia. En las instituciones p\'ublicas, universidad Nacional, Distrital y Pedag\'ogica, se tiene m\'as conocimiento sobre la existencia del software libre, siendo la Universidad Javeriana una instituci\'on privada. Siendo la Universidad Nacional la que tiene m\'as conocimiento sobre la existencia del SL seguida de cerca por la Universidad Distrital.\\

Sobre el \'item 2, ¿utiliza software libre?, es una pregunta abierta, donde el encuestado responde de acuerdo a su criterio en referencia a la utilizaci\'on.\\

\begin{center}
\resizebox{8.1cm}{!}{{\epsfbox{images/Conocimiento/fig2.eps}}}
\mycaption{Respuestas a la pregunta ¿utiliza software libre? }
\end{center}


En la mayor\'ia de las instituciones, los estudiantes respondieron que NO usan software libre, excepto en el SENA donde predomina el NS/NA.\\

La Universidad Nacional de Colombia es la instituci\'on d\'onde m\'as porcentaje de sus estudiantes usan software libre. Esto est\'a en concordancia con el \'item 1. Donde dicha instituci\'on tiene mas conocimiento sobre la existencia del SL.\\

En el \'item 3, del ¿por qu\'e dejo de usar software libre?, se les colocó cinco alternativas obvias, sin que se presentara otra opción dada por el encuestado.\\

\begin{center}
\resizebox{8cm}{!}{{\epsfbox{images/Conocimiento/fig3.eps}}}
\mycaption{Estadísticas ¿por qu\'e dejo de usar software libre?}
\end{center}

Dejando a un lado el NS/NA ya que es una respuesta de los estudiantes que no conocen el SL, en todas las instituciones predomina el “a\'un lo utilizo”, es decir que de los estudiantes que saben de software libre a\'un lo siguen utiliz\'andolo. Luego, el desuso por virus, trabajo, o porque es complicado tienen un porcentaje muy similar, en especial entre los aspectos de “Por los virus” y “Porque es complicado”. \\

Ac\'a hay un fen\'omeno particular, y bien diferenciado, que la Universidad Distrital y la Universidad Pedag\'ogica lo utilizan en un porcentaje mayor de 10 puntos que la siguiente instituci\'on que es la Universidad Nacional, siendo relegadas en la utilizaci\'on del SL la instituci\'on privada, Universidad Javeriana y muy por debajo el SENA.\\

Con el \'item 4, ¿si en la instituci\'on hubieran cursos de SL, los tomaria?, al encuestado se le dejo una opción posible ademas del SI y el NO. \\

% Siguiente página
\ebOpage{introcolor}{0.35}{Utilización Software Libre}

\begin{center}
\resizebox{8cm}{!}{{\epsfbox{images/Conocimiento/fig4.eps}}}
\mycaption{Respuestas acerca de cursos sobre Software Libre}
\end{center}

Se muestra que los estudiantes encuestados tomar\'ian cursos acerca de software libre, o tal vez lo tomarían, donde estas dos alternativas son contundentes y atractivas para los estudiantes. Esto muestra que los estudiantes no dudan de su interés.\\

Es de notar que la Universidad Nacional es poco entusiasta en referencia a tomar cursos de SL y  La universidad Javeriana es la que tiene menor interés en estas herramientas. Sin embargo las entidades que muestran bajo conocimiento en este tipo de software, tienen altos niveles de interés en aprenderlas.\\

Los ítems del 5 al 10 son de tipo binario, en la cual se hace un bosquejo sobre conceptos generales, para indagar el grado de conocimiento que los estudiantes tiene sobre la filosof\'ia del SL.\\

Dado el crecimiento de las aplicaciones de SL en las diferentes áreas de la educación, se inicio con el ítem 5, el cual se pregunto: ¿La oferta de programas es limitada?. para este ítem, el estudio arrojo la información mostrada en la figura.\\

\begin{center}
\resizebox{8cm}{!}{{\epsfbox{images/Conocimiento/fig5.eps}}}
\mycaption{¿La oferta de programas es limitada?}
\end{center}

Se puede resaltar que los estudiantes encuestados no saben de la variedad de software libre que hay, en promedio el 25 porciento tienen certeza, donde las universidades públicas, y en especial la Universidad Nacional,  encabezan las instituciones que conocen de la variedad de herramientas de SL, las cuales están presentes en todas las áreas del conocimiento.\\

Para el \'item 6, ¿Es posible utilizar programas de software libre en Windows y Mac?, para el cual se obtuvo la siguiente información, mostrada en la figura.\\

\begin{center}
\resizebox{8cm}{!}{{\epsfbox{images/Conocimiento/fig7.eps}}}
\mycaption{Respuestas acerca del conocimiento de Software Libre en sistemas privativos.}
\end{center}

Se basa en el principio de que el software es de c\'odigo abierto y por lo tanto se puede adaptar a cualquier sistema. Se puede observar en la figura que la Universidad Nacional tienen m\'as claridad sobre este \'item, siendo la Universidad Javeriana la instituci\'on. \\

El \'item 6, ¿Copiar y redistribuir software libre es legal?, es otro principio del SL, luego es legal en este aspecto. Aun cuando en la mayor\'ia de las instituciones, la alternativa “NO SE” predomina, es claro que la otra alternativa “SI” conoce este principio sobre la distribuci\'on del SL y en mayor proporci\'on La Universidad Distrital.\\
En el \'item 7, ¿El software libre puede ser comercial?

% Siguiente página
\ebOpage{introcolor}{0.35}{Utilización Software Libre}

\begin{center}
\resizebox{8cm}{!}{{\epsfbox{images/Conocimiento/fig8.eps}}}
\mycaption{Conocimiento acerca de la comercialización del Software Libre}
\end{center}

No est\'a en los principios fundamentales para definir el SL, luego es posible que el software libre se comercial y sea catalogado como SL. Por esto la Universidad Distrital tiene m\'as certeza sobre este \'item, aun cuando en mayor grado los estudiantes no lo determinan y la alternativa “NO SE” es mayoritaria.\\

Sobre el \'item 8, ¿El software libre es gratis?, es muy dado a confundir lo libre con lo gratuito, dado que la palabra anglosajona ``free'' cubre estos dos aspectos.\\

\begin{center}
\resizebox{8cm}{!}{{\epsfbox{images/Conocimiento/fig9.eps}}}
\mycaption{Respuestas de la relación entre libertad y gratis}
\end{center}

Aunque la mayor\'ia del SL es gratis, no es un principio en la filosof\'ia del software libre, ya que hay software gratis y no es libre puesto que no se tiene acceso al c\'odigo fuente, tambi\'en hay SL que tiene costo y cumple con los principios para ser catalogado como SL. Este \'item muestra un desconocimiento en lo que hace referencia al concepto de gratuidad y SL. Aun cuando la pregunta puede confundir, puesto que en el com\'un se asocia a SL con gratuidad.\\

El \'item 6, ¿Copiar y redistribuir software libre es legal?, es muy evidente la alternativa dominante. \\

\begin{center}
\resizebox{8cm}{!}{{\epsfbox{images/Conocimiento/fig6.eps}}}
\mycaption{Respuestas acerca de la legalidad al copiar y redistribuir software libre}
\end{center}


Es otro principio del SL, luego es legal en este aspecto. Aún cuando en la mayor\'ia de las instituciones, la alternativa “NO SE” predomina, es claro que la otra alternativa “SI” conoce este principio sobre la distribuci\'on del SL y en mayor proporci\'on La Universidad Distrital.\\

\begin{center}
\resizebox{8cm}{!}{{\epsfbox{images/Conocimiento/fig10.eps}}}
\mycaption{Derechos de autor en el Software Libre}
\end{center}

\sectiontext{white}{black}{Utilización en los segmentos del software libre}

En este apartado, se quiso indagar por las preferencias en la utilización, del software en general, de segmentos de programas utilizados en los computadores personales de los estudiantes encuestados,

% Siguiente página
\ebOpage{introcolor}{0.35}{Utilización Software Libre}

Los cuales están relacionados con el sistema operativo, paquetes de ofimática, de edición de imágenes y de vídeo, y navegadores. Se selecciono el software de cada segmento que mas se menciona en la clase universitaria. Se incluyo software comercial, ya que por tradición es el que se ha utilizado en los últimos tiempos.\\

En el ítem 11. se les indago a los encuestados el tipo de sistema operativo que está instalado en el computador, a lo cual los resultados se muestran en la figura 11.  se nota que el sistema operativo dominante es el Windows y es el m\'as utilizado en general en las instituciones. \\

\begin{center}
\resizebox{8cm}{!}{{\epsfbox{images/Conocimiento/fig11.eps}}}
\mycaption{Sistemas operativos utilizados actualmente}
\end{center}

Cabe destacar tambi\'en que un porcentaje considerable utiliza el  Sistema Operativo Mac y en especial la entidad  Pontificia Universidad Javeriana, una de las razones consideradas hace referencia al nivel socio-económicos de los encuestados. El fenómeno  del 10.6 \% de la utilización del software en la entidad pública de la universidad Nacional, indica que  estudiantes de niveles socio-económicos importantes están optando por esta institución.\\

El segmento de navegadores, figura 12,  muestra en este ítem a chrome como el navegador m\'as utilizado en general, por muy encima de Firefox que es el segundo m\'as utilizado a excepción de la Universidad Javeriana que, concuerda con el item 11 dada la utilización del sistema operativo MAC, tienen en segundo lugar el navegador safari.

\begin{center}
\resizebox{8cm}{!}{{\epsfbox{images/Conocimiento/fig12.eps}}}
\mycaption{Navegadores utilizados actualmente}
\end{center}

Es interesante ver el repunte de los navegadores de software libre, dado que los comerciales están muy diezmados en su uso.\\
En cuanto a paquetes de software de ofimática, el software propietario de Microsoft sigue repuntando muy lejos respecto a los de uso libre.

\begin{center}
\resizebox{8cm}{!}{{\epsfbox{images/Conocimiento/fig13.eps}}}
\mycaption{Software de ofim\'atica utilizado actualmente}
\end{center}

El software de ofim\'atica m\'as utilizado por los encuestados en las diferentes instituciones es Microsoft Office, pero esto muestra que las porcentajes de uso están bajando levemente dada las bondades de software que aparece en el escenario como lo es Google docs, que es un aplicativo de la nube, y los paquetes de software libre que están en niveles similares. \\
Y se observa que Google docs, Libre Office y Open Office tienen porcentajes similares en su  uso.\\

% Siguiente página
\ebOpage{introcolor}{0.35}{Utilización Software Libre}

El segmento del ítem 13, como es la multimedia, es uno de los mas diversos en su utilización, y dependiendo de la institución predomina una aplicación, donde Itunes, VLC  y windows media players (WMP) repuntas en diferentes instituciones, donde WMP predomina en la Universidad Distrital Francisco Jos\'e de Caldas, al igual que en el SENA y en la Universidad Pedag\'ogica Nacional.\\

\begin{center}
\resizebox{8cm}{!}{{\epsfbox{images/Conocimiento/fig14.eps}}}
\mycaption{Utilización actual de multimedia}
\end{center}

Hay diversidad en este segmento, dado que los porcentajes de utilización en la alternativa “otros” es considerable, esto muestra que para próximos estudios estas alternativas deben ampliarse a mas programas de este tipo.\\

Al igual que en el software de multimedia para el manejo de imágenes, figura 15,  tambi\'en vemos una variedad de uso en programas de visualizaci\'on de im\'agenes. Sin embargo el software propietario de la marca Microsoft predomina con sus dos paquetes, paint y MS office picture manager, los cuales son dominantes en este segmento.\\

\begin{center}
\resizebox{8cm}{!}{{\epsfbox{images/Conocimiento/fig15.eps}}}
\mycaption{Visor de imágenes utilizados actualmente}
\end{center}

Para este ítem, el software libre se mantiene en niveles aceptables en la tabla, figura 15, y es consecuente con los demás ítems de la encuesta. \\

\sectiontext{white}{black}{Análisis global}

De manera general, sobre el conocimiento del SL en las instituciones educativas, el no haber oído mencionarlo da curiosidad, dado que el software libre está orientado a la academia y a la investigación, siendo los principales objetivos misionales de las universidades.\\

Los usuarios de software libre que se han iniciado en su uso, al rededor del 50\% lo dejaron de utilizar y el principal factor que aducen es que es complicado, figura 3. sin embargo, es esperanzador que el 25\% de los que conocen de software libre aún lo utilizan. Y otro punto a resaltar es que el encuestado tiene interés en aprenderlo, si hubiera cursos formales en las instituciones, donde solo el 7.5\% no tendría interés.\\

En cuanto a la utilización de herramientas software de uso personal, se evidencia que los sistemas operativos y los de ofimática, están siendo dominados por el software privativo de la casa Microsoft, sin embargo, su hegemonía ha venido bajando por la presencia creciente de los paquetes de SL y de la nube como lo es Google docs. Las herramientas de multimedia y navegadores, los ofrecidos en el SL están posicionados entre los encuestados.\\

\sectiontext{white}{black}{V. Conclusiones}

El estudio muestra que en las instituciones universitaria no se incentiva el uso de herramientas de software libre, a sabiendas de las ventajas que tiene sobre el software privativo.\\

Se recomienda a las instituciones ofrecer cursos para el aprendizaje de herramientas de SL ligadas la academia, dado que los estudiantes están dispuestos, en su mayoría, a tomar estos cursos.\\

 Las políticas públicas permiten a las instituciones incentivar en la utilización del software libre como un mecanismo de austeridad en el gasto en lo referente al software. Sin olvidar las ventajas sobre los principios de libertad en el software, que es el principal valor.\\
 
Las comunidades de software libre tienen una gran oportunidad en las instituciones en lo referente a la difusión, enseñanza y generación de productos de software, no solo a la academia sino a la sociedad en general.
\\
% Siguiente página
\ebOpage{introcolor}{0.35}{Utilización Software Libre}

\bibliographystyle{abbrv}
\begin{bibliografia}
\bibitem{kopka}
Stallman, Richard. \emph{Software Libre para una Sociedad Libre}, \href{https://www.gnu.org/philosophy/fsfs/free_software.es.pdf}{disponible en pdf}.\hskip 1em plus
0.5em minus 0.4em\relax Madrid, España: Traficantes de sueños, 2004.

\end{bibliografia}

\begin{biografia}{images/general/white.eps}{José Noé Poveda} 
Msg. en Teleinformática, docente de la Facultad de Ingeniería Universidad Distrital Francisco José de Caldas, actual coordinador del Grupo GNU Linux Universidad Distrital.
\end{biografia}

\raggedcolumns
\pagebreak

\end{multicols}

\clearpage
\pagebreak